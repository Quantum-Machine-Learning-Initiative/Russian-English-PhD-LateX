%%% Основные сведения %%%
\newcommand{\thesisAuthorLastName}{\fixme{Фамилия}}
\newcommand{\thesisAuthorOtherNames}{\fixme{Имя Отчество}}
\newcommand{\thesisAuthorInitials}{\fixme{И.\,О.}}
\newcommand{\thesisAuthorLastNameEn}{\fixme{Lastname}}
\newcommand{\thesisAuthorOtherNamesEn}{\fixme{Firstname Middlename}}
\newcommand{\thesisAuthorInitialsEn}{\fixme{F.\,M.}}

\newcommand{\thesisAuthor}             % Диссертация, ФИО автора
{%
    \texorpdfstring{% \texorpdfstring takes two arguments and uses the first for (La)TeX and the second for pdf
        \thesisAuthorLastName~\thesisAuthorOtherNames% так будет отображаться на титульном листе или в тексте, где будет использоваться переменная
    }{%
        \thesisAuthorLastName, \thesisAuthorOtherNames% эта запись для свойств pdf-файла. В таком виде, если pdf будет обработан программами для сбора библиографических сведений, будет правильно представлена фамилия.
    }
}
\newcommand{\thesisAuthorEn}             % Диссертация, ФИО автора
{%
    \texorpdfstring{% \texorpdfstring takes two arguments and uses the first for (La)TeX and the second for pdf
        \thesisAuthorLastNameEn~\thesisAuthorOtherNamesEn% так будет отображаться на титульном листе или в тексте, где будет использоваться переменная
    }{%
        \thesisAuthorLastNameEn, \thesisAuthorOtherNamesEn% эта запись для свойств pdf-файла. В таком виде, если pdf будет обработан программами для сбора библиографических сведений, будет правильно представлена фамилия.
    }
}

\newcommand{\thesisAuthorShort}        % Диссертация, ФИО автора инициалами
{\thesisAuthorInitials~\thesisAuthorLastName}
\newcommand{\thesisAuthorShortEn}        % Диссертация, ФИО автора инициалами
{\thesisAuthorInitialsEn~\thesisAuthorLastNameEn}

%\newcommand{\thesisUdk}                % Диссертация, УДК
%{\fixme{xxx.xxx}}
\newcommand{\thesisTitle}              % Диссертация, название
{\fixme{Длинное название диссертационной работы, состоящее из~достаточно большого
количества слов, совсем длинное длинное длинное длинное название, из~которого
простому обывателю знакомы, в~лучшем случае, лишь отдельные слова}}
\newcommand{\thesisTitleEn}              % Диссертация, название
{\fixme{A very long title of the dissertation, with many many many many many many many many many many many many words, most of which are completely opaque to a nonspecialist}}
\newcommand{\thesisSpecialtyNumber}    % Диссертация, специальность, номер
{\fixme{XX.XX.XX}}
\newcommand{\thesisSpecialtyTitle}     % Диссертация, специальность, название (название взято с сайта ВАК для примера)
{\fixme{Технология обработки, хранения и~переработки злаковых, бобовых культур,
крупяных продуктов, плодоовощной продукции и~виноградарства}}
\newcommand{\thesisSpecialtyTitleEn}     % Диссертация, специальность, название (название взято с сайта ВАК для примера)
{\fixme{Theoretical physics}}
%% \newcommand{\thesisSpecialtyTwoNumber} % Диссертация, вторая специальность, номер
%% {\fixme{XX.XX.XX}}
%% \newcommand{\thesisSpecialtyTwoTitle}  % Диссертация, вторая специальность, название
%% {\fixme{Теория и~методика физического воспитания, спортивной тренировки,
%% оздоровительной и~адаптивной физической культуры}}
\newcommand{\thesisDegree}             % Диссертация, ученая степень
{\fixme{кандидата физико-математических наук}}
\newcommand{\thesisDegreeEn}             % Диссертация, ученая степень
{\fixme{candidate of physical and mathematical sciences}}
\newcommand{\thesisDegreeShort}        % Диссертация, ученая степень, краткая запись
{\fixme{канд. физ.-мат. наук}}
\newcommand{\thesisDegreeShortEn}        % Диссертация, ученая степень, краткая запись
{\fixme{cand. of phys.-math. sciences}}
\newcommand{\thesisCity}               % Диссертация, город написания диссертации
{\fixme{Город}}
\newcommand{\thesisCityEn}               % Диссертация, город написания диссертации
{\fixme{City}}
\newcommand{\thesisYear}               % Диссертация, год написания диссертации
{\the\year}
\newcommand{\thesisOrganization}       % Диссертация, организация
{\fixme{Федеральное государственное автономное образовательное учреждение высшего
образования <<Длинное название образовательного учреждения <<АББРЕВИАТУРА>>}}
\newcommand{\thesisOrganizationEn}       % Диссертация, организация
{\fixme{Federal state autonomous organization for higher education ``Long title of the institute <<ACRONYM>>''}}
\newcommand{\thesisOrganizationShort}  % Диссертация, краткое название организации для доклада
{\fixme{НазУчДисРаб}}
\newcommand{\thesisOrganizationShortEn}  % Диссертация, краткое название организации для доклада
{\fixme{ThesOrgShortTtl}}

\newcommand{\thesisInOrganization}     % Диссертация, организация в предложном падеже: Работа выполнена в ...
{\fixme{учреждении с~длинным длинным длинным длинным названием, в~котором
выполнялась данная диссертационная работа}}

%% \newcommand{\supervisorDead}{}           % Рисовать рамку вокруг фамилии
\newcommand{\supervisorFio}              % Научный руководитель, ФИО
{\fixme{Фамилия Имя Отчество}}
\newcommand{\supervisorRegalia}          % Научный руководитель, регалии
{\fixme{уч. степень, уч. звание}}
\newcommand{\supervisorFioShort}         % Научный руководитель, ФИО
{\fixme{И.\,О.~Фамилия}}
\newcommand{\supervisorRegaliaShort}     % Научный руководитель, регалии
{\fixme{уч.~ст.,~уч.~зв.}}

\newcommand{\supervisorFioEn}              % Научный руководитель, ФИО
{\fixme{Lastname Firstname Middlename}}
\newcommand{\supervisorRegaliaEn}          % Научный руководитель, регалии
{\fixme{Doctor of philosophy, Professor}}
\newcommand{\supervisorFioShortEn}         % Научный руководитель, ФИО
{\fixme{F.\,M.~Lastname}}
\newcommand{\supervisorRegaliaShortEn}     % Научный руководитель, регалии
{\fixme{PhD, Prof}}

%% \newcommand{\supervisorTwoDead}{}        % Рисовать рамку вокруг фамилии
%% \newcommand{\supervisorTwoFio}           % Второй научный руководитель, ФИО
%% {\fixme{Фамилия Имя Отчество}}
%% \newcommand{\supervisorTwoRegalia}       % Второй научный руководитель, регалии
%% {\fixme{уч. степень, уч. звание}}
%% \newcommand{\supervisorTwoFioShort}      % Второй научный руководитель, ФИО
%% {\fixme{И.\,О.~Фамилия}}
%% \newcommand{\supervisorTwoRegaliaShort}  % Второй научный руководитель, регалии
%% {\fixme{уч.~ст.,~уч.~зв.}}

\newcommand{\opponentOneFio}           % Оппонент 1, ФИО
{\fixme{Фамилия Имя Отчество}}
\newcommand{\opponentOneRegalia}       % Оппонент 1, регалии
{\fixme{доктор физико-математических наук, профессор}}
\newcommand{\opponentOneJobPlace}      % Оппонент 1, место работы
{\fixme{Не очень длинное название для места работы}}
\newcommand{\opponentOneJobPost}       % Оппонент 1, должность
{\fixme{старший научный сотрудник}}

\newcommand{\opponentTwoFio}           % Оппонент 2, ФИО
{\fixme{Фамилия Имя Отчество}}
\newcommand{\opponentTwoRegalia}       % Оппонент 2, регалии
{\fixme{кандидат физико-математических наук}}
\newcommand{\opponentTwoJobPlace}      % Оппонент 2, место работы
{\fixme{Основное место работы c длинным длинным длинным длинным названием}}
\newcommand{\opponentTwoJobPost}       % Оппонент 2, должность
{\fixme{старший научный сотрудник}}

%% \newcommand{\opponentThreeFio}         % Оппонент 3, ФИО
%% {\fixme{Фамилия Имя Отчество}}
%% \newcommand{\opponentThreeRegalia}     % Оппонент 3, регалии
%% {\fixme{кандидат физико-математических наук}}
%% \newcommand{\opponentThreeJobPlace}    % Оппонент 3, место работы
%% {\fixme{Основное место работы c длинным длинным длинным длинным названием}}
%% \newcommand{\opponentThreeJobPost}     % Оппонент 3, должность
%% {\fixme{старший научный сотрудник}}

\newcommand{\leadingOrganizationTitle} % Ведущая организация, дополнительные строки. Удалить, чтобы не отображать в автореферате
{\fixme{Федеральное государственное бюджетное образовательное учреждение высшего
профессионального образования с~длинным длинным длинным длинным названием}}
\newcommand{\leadingOrganizationTitleEn} % Ведущая организация, дополнительные строки. Удалить, чтобы не отображать в автореферате
{\fixme{Federal state organization tratatatatata}}

\newcommand{\defenseDate}              % Защита, дата
{\fixme{DD mmmmmmmm YYYY~г.~в~XX часов}}
\newcommand{\defenseDateEn}              % Защита, дата
{\fixme{DD mmmmmmmm YYYY~at~XX:YY}}
\newcommand{\defenseCouncilNumber}     % Защита, номер диссертационного совета
{\fixme{Д\,123.456.78}}
\newcommand{\defenseCouncilTitle}      % Защита, учреждение диссертационного совета
{\fixme{Название учреждения}}
\newcommand{\defenseCouncilTitleEn}      % Защита, учреждение диссертационного совета
{\fixme{Defence council title}}
\newcommand{\defenseCouncilAddress}    % Защита, адрес учреждение диссертационного совета
{\fixme{Адрес}}
\newcommand{\defenseCouncilAddressEn}    % Защита, адрес учреждение диссертационного совета
{\fixme{Address}}
\newcommand{\defenseCouncilPhone}      % Телефон для справок
{\fixme{+7~(0000)~00-00-00}}

\newcommand{\defenseSecretaryFio}      % Секретарь диссертационного совета, ФИО
{\fixme{Фамилия Имя Отчество}}
\newcommand{\defenseSecretaryRegalia}  % Секретарь диссертационного совета, регалии
{\fixme{д-р~физ.-мат. наук}}            % Для сокращений есть ГОСТы, например: ГОСТ Р 7.0.12-2011 + http://base.garant.ru/179724/#block_30000

\newcommand{\defenseSecretaryFioEn}      % Секретарь диссертационного совета, ФИО
{\fixme{Lastname First Middle}}
\newcommand{\defenseSecretaryRegaliaEn}  % Секретарь диссертационного совета, регалии
{\fixme{DSc.}}    

\newcommand{\synopsisLibrary}          % Автореферат, название библиотеки
{\fixme{Название библиотеки}}
\newcommand{\synopsisDate}             % Автореферат, дата рассылки
{\fixme{DD mmmmmmmm}\the\year~года}

\newcommand{\synopsisLibraryEn}          % Автореферат, название библиотеки
{\fixme{Library title}}
\newcommand{\synopsisDateEn}             % Автореферат, дата рассылки
{\fixme{DD mmmmmmmm}\the\year}

% To avoid conflict with beamer class use \providecommand
\providecommand{\keywords}%            % Ключевые слова для метаданных PDF диссертации и автореферата
{}
